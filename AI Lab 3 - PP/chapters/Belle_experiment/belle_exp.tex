\section{The Belle experiment}

The Belle experiment is a so-called \emph{B-Factory}, 
a type of particle accelerator specifically designed to 
produce large numbers of \emph{B mesons}. 
These factories operate with a center-of-mass energy tuned 
to the \emph{$\Upsilon(4S)$ resonance}, 
a state that decays almost exclusively 
into pairs of $B$ and $\bar{B}$ mesons. 
This production mechanism enables detailed studies 
of $B$ meson decays, allowing physicists to investigate 
phenomena such as \emph{CP violation}, rare decays, 
and physics beyond the Standard Model. 
The Belle~II experiment, located at the \emph{SuperKEKB collider in 
Tsukuba, Japan}, is the successor to the original 
Belle experiment and significantly improves on its 
precision and data collection capabilities.

Unlike proton-proton colliders (such as the LHC), Belle~II uses 
electron-positron ($e^+e^-$) collisions. Electrons and positrons 
are elementary particles with no internal structure, this offers 
several advantages. First, their interactions are much cleaner, and 
the background is easier to interpret—there is a notable absence of 
hadronic activity such as jets. Moreover, the absence of 
internal substructure allows us to precisely know the initial momenta 
of the colliding particles, enabling accurate reconstruction of the 
event kinematics. This precision is crucial for tuning the  
aforementioned center-of-mass energy to specific resonances, 
such as the $\Upsilon(4S)$.

$B$ mesons are unstable particles that decay via the 
weak interaction, and their decays can be classified into 
three categories: hadronic, semileptonic, and rare  decays. 
In \emph{hadronic decays}, the $B$ meson decays into only 
hadrons, such as combinations of kaons ($K$), pions ($\pi$), 
and other mesons. \emph{Semileptonic decays} involve 
both hadrons and leptons, typically resulting in a final 
state containing a charged lepton ($e^\pm$ or $\mu^\pm$), 
a neutrino, and hadrons. \emph{Rare decays} may involve photons,
multiple leptons, or other suppressed processes that are 
especially sensitive to effects beyond the Standard Model.
These decay modes provide a rich ground for studying CP 
violation, flavor physics, and potential new physics.

The Belle~II detector is structured as a layered system of 
subdetectors (like an onion) built around the interaction point.
Closest to the interaction region is the Vertex Detector (VXD), 
composed of the PiXel Detector (PXD) and Silicon 
Vertex Detector (SVD), which provide precise tracking 
information for reconstructing decay vertices. 
Surrounding the VXD is the Central Drift Chamber (CDC), 
responsible for tracking charged particles and measuring 
their momenta and energy loss. 
Particle identification is handled by the Time Of 
Propagation (TOP) detector in the barrel region and the 
Aerogel Ring Imaging Cherenkov (ARICH) detector in the 
forward region.Photons and electrons are measured in the 
Electromagnetic Calorimeter (ECL) via the production of showers,
while muons and long-lived neutral kaons ($K_L^0$) are 
identified in the outermost KLong and Muon (KLM) detector. 
The entire setup is enclosed in a $1.5~T$ magnetic field provided 
by a superconducting solenoid, which not only alligns the beams
but also enables accurate momentum measurements. An overview 
of the collider (figure~\ref{fig:belle-detector-overview} ) and the particles that can be identified in each 
subdetector can be seen in table~\ref{table:belle-subdetectors}


\begin{figure}[h]
    \centering
    \includegraphics[width=0.8\linewidth]{chapters/Belle_experiment/Overview_collider.png}
    \caption{Closeup of the Belle~II detector indicating 
    all the different subdetectors~\cite{belle2-image}.}
    \label{fig:belle-detector-overview}
\end{figure}


\begin{table}[h]
\centering
\begin{tabular}{ll}
\toprule
\textbf{Subdetector} & \textbf{Main Purpose} \\
\midrule
PXD + SVD (VXD) & Track reconstruction near the IP, measures short-lived particles \\
CDC             & Tracking of charged particles \\
TOP / ARICH     & Particle identification (e.g. distinguish $\pi$, $K$, $p$) via Cherenkov radiation \\
ECL             & Detection of electrons and photons (electromagnetic showers) \\
KLM             & $\mu$ identification and detection of $K_L^0$ \\
\bottomrule
\end{tabular}
\caption{Belle~II subdetectors and the particles they primarily detect.}
\label{table:belle-subdetectors}
\end{table}


\subsection{Problem Description}

The Belle~II experiment collects an enormous amount of data,
with its center-of-mass energy tuned to the $\Upsilon(4S)$ resonance 
\emph{corresponding to a center-of-mass energy of $\sqrt{s} = 10.58$~GeV}. 
While these "resonant" events are the primary source 
for $B$ physics, Belle~II also records other types 
of data.

To achieve this center-of-mass energy the two beams colliding have respective
energies of $7\, \text{GeV}$ and $4\, \text{GeV}$. This assymetric beam 
configuration introduces a forward boost to the CoM system, forcing the 
produced $B$ (and $\bar{B}$) mesons to travel measurables distances
in the detector before they decay.

\begin{figure}[h]
    \centering
    \includegraphics[width=0.75\linewidth]{chapters/Belle_experiment/Belle_resonances.png}
    \caption{Cross section of $e^+e^-$ with respect to the center-of-mass energy [GeV] \cite{belle2-Y-resonances}.}
    \label{fig:upsilon-resonances}
\end{figure}

To reduce the data volume and focus on interesting events, 
Belle~II employs a two-level filtering system: 
the Level-1 hardware trigger (TRG) and the High Level Trigger (HLT), 
which together reduce the event rate before data is stored. 
Even after this, the number of events retained is still too large 
for direct analysis, so further selection — known as skimming — is 
performed to extract smaller datasets.

In parallel, Monte Carlo (MC) simulations are used to model the 
detector response and to train and validate analysis strategies.
These simulated events must pass through all stages of detector 
simulation, digitization, reconstruction, and selection, mirroring 
the real data processing pipeline. However,this process is 
computationally expensive, especially when simulating large 
background samples, most of which are eventually discarded by skims.

To address this issue, the goal of this lab course is to
explore a technique known as \emph{Smart Background Simulation}. 
Instead of fully simulating every event, we aim to train a neural 
network that can predict early on whether a given event will survive a 
later skim. This skim selects events in which at least one $B^0$ meson 
can be reconstructed through hadronic decays. 
After building several neural networks we evaluate their performance
on an independent test set and calculate the speedup offered by each network.






